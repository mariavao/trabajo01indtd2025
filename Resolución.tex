% Options for packages loaded elsewhere
\PassOptionsToPackage{unicode}{hyperref}
\PassOptionsToPackage{hyphens}{url}
\PassOptionsToPackage{dvipsnames,svgnames,x11names}{xcolor}
%
\documentclass[
  letterpaper,
  DIV=11,
  numbers=noendperiod]{scrartcl}

\usepackage{amsmath,amssymb}
\usepackage{iftex}
\ifPDFTeX
  \usepackage[T1]{fontenc}
  \usepackage[utf8]{inputenc}
  \usepackage{textcomp} % provide euro and other symbols
\else % if luatex or xetex
  \usepackage{unicode-math}
  \defaultfontfeatures{Scale=MatchLowercase}
  \defaultfontfeatures[\rmfamily]{Ligatures=TeX,Scale=1}
\fi
\usepackage{lmodern}
\ifPDFTeX\else  
    % xetex/luatex font selection
\fi
% Use upquote if available, for straight quotes in verbatim environments
\IfFileExists{upquote.sty}{\usepackage{upquote}}{}
\IfFileExists{microtype.sty}{% use microtype if available
  \usepackage[]{microtype}
  \UseMicrotypeSet[protrusion]{basicmath} % disable protrusion for tt fonts
}{}
\makeatletter
\@ifundefined{KOMAClassName}{% if non-KOMA class
  \IfFileExists{parskip.sty}{%
    \usepackage{parskip}
  }{% else
    \setlength{\parindent}{0pt}
    \setlength{\parskip}{6pt plus 2pt minus 1pt}}
}{% if KOMA class
  \KOMAoptions{parskip=half}}
\makeatother
\usepackage{xcolor}
\setlength{\emergencystretch}{3em} % prevent overfull lines
\setcounter{secnumdepth}{5}
% Make \paragraph and \subparagraph free-standing
\ifx\paragraph\undefined\else
  \let\oldparagraph\paragraph
  \renewcommand{\paragraph}[1]{\oldparagraph{#1}\mbox{}}
\fi
\ifx\subparagraph\undefined\else
  \let\oldsubparagraph\subparagraph
  \renewcommand{\subparagraph}[1]{\oldsubparagraph{#1}\mbox{}}
\fi

\usepackage{color}
\usepackage{fancyvrb}
\newcommand{\VerbBar}{|}
\newcommand{\VERB}{\Verb[commandchars=\\\{\}]}
\DefineVerbatimEnvironment{Highlighting}{Verbatim}{commandchars=\\\{\}}
% Add ',fontsize=\small' for more characters per line
\usepackage{framed}
\definecolor{shadecolor}{RGB}{241,243,245}
\newenvironment{Shaded}{\begin{snugshade}}{\end{snugshade}}
\newcommand{\AlertTok}[1]{\textcolor[rgb]{0.68,0.00,0.00}{#1}}
\newcommand{\AnnotationTok}[1]{\textcolor[rgb]{0.37,0.37,0.37}{#1}}
\newcommand{\AttributeTok}[1]{\textcolor[rgb]{0.40,0.45,0.13}{#1}}
\newcommand{\BaseNTok}[1]{\textcolor[rgb]{0.68,0.00,0.00}{#1}}
\newcommand{\BuiltInTok}[1]{\textcolor[rgb]{0.00,0.23,0.31}{#1}}
\newcommand{\CharTok}[1]{\textcolor[rgb]{0.13,0.47,0.30}{#1}}
\newcommand{\CommentTok}[1]{\textcolor[rgb]{0.37,0.37,0.37}{#1}}
\newcommand{\CommentVarTok}[1]{\textcolor[rgb]{0.37,0.37,0.37}{\textit{#1}}}
\newcommand{\ConstantTok}[1]{\textcolor[rgb]{0.56,0.35,0.01}{#1}}
\newcommand{\ControlFlowTok}[1]{\textcolor[rgb]{0.00,0.23,0.31}{#1}}
\newcommand{\DataTypeTok}[1]{\textcolor[rgb]{0.68,0.00,0.00}{#1}}
\newcommand{\DecValTok}[1]{\textcolor[rgb]{0.68,0.00,0.00}{#1}}
\newcommand{\DocumentationTok}[1]{\textcolor[rgb]{0.37,0.37,0.37}{\textit{#1}}}
\newcommand{\ErrorTok}[1]{\textcolor[rgb]{0.68,0.00,0.00}{#1}}
\newcommand{\ExtensionTok}[1]{\textcolor[rgb]{0.00,0.23,0.31}{#1}}
\newcommand{\FloatTok}[1]{\textcolor[rgb]{0.68,0.00,0.00}{#1}}
\newcommand{\FunctionTok}[1]{\textcolor[rgb]{0.28,0.35,0.67}{#1}}
\newcommand{\ImportTok}[1]{\textcolor[rgb]{0.00,0.46,0.62}{#1}}
\newcommand{\InformationTok}[1]{\textcolor[rgb]{0.37,0.37,0.37}{#1}}
\newcommand{\KeywordTok}[1]{\textcolor[rgb]{0.00,0.23,0.31}{#1}}
\newcommand{\NormalTok}[1]{\textcolor[rgb]{0.00,0.23,0.31}{#1}}
\newcommand{\OperatorTok}[1]{\textcolor[rgb]{0.37,0.37,0.37}{#1}}
\newcommand{\OtherTok}[1]{\textcolor[rgb]{0.00,0.23,0.31}{#1}}
\newcommand{\PreprocessorTok}[1]{\textcolor[rgb]{0.68,0.00,0.00}{#1}}
\newcommand{\RegionMarkerTok}[1]{\textcolor[rgb]{0.00,0.23,0.31}{#1}}
\newcommand{\SpecialCharTok}[1]{\textcolor[rgb]{0.37,0.37,0.37}{#1}}
\newcommand{\SpecialStringTok}[1]{\textcolor[rgb]{0.13,0.47,0.30}{#1}}
\newcommand{\StringTok}[1]{\textcolor[rgb]{0.13,0.47,0.30}{#1}}
\newcommand{\VariableTok}[1]{\textcolor[rgb]{0.07,0.07,0.07}{#1}}
\newcommand{\VerbatimStringTok}[1]{\textcolor[rgb]{0.13,0.47,0.30}{#1}}
\newcommand{\WarningTok}[1]{\textcolor[rgb]{0.37,0.37,0.37}{\textit{#1}}}

\providecommand{\tightlist}{%
  \setlength{\itemsep}{0pt}\setlength{\parskip}{0pt}}\usepackage{longtable,booktabs,array}
\usepackage{calc} % for calculating minipage widths
% Correct order of tables after \paragraph or \subparagraph
\usepackage{etoolbox}
\makeatletter
\patchcmd\longtable{\par}{\if@noskipsec\mbox{}\fi\par}{}{}
\makeatother
% Allow footnotes in longtable head/foot
\IfFileExists{footnotehyper.sty}{\usepackage{footnotehyper}}{\usepackage{footnote}}
\makesavenoteenv{longtable}
\usepackage{graphicx}
\makeatletter
\def\maxwidth{\ifdim\Gin@nat@width>\linewidth\linewidth\else\Gin@nat@width\fi}
\def\maxheight{\ifdim\Gin@nat@height>\textheight\textheight\else\Gin@nat@height\fi}
\makeatother
% Scale images if necessary, so that they will not overflow the page
% margins by default, and it is still possible to overwrite the defaults
% using explicit options in \includegraphics[width, height, ...]{}
\setkeys{Gin}{width=\maxwidth,height=\maxheight,keepaspectratio}
% Set default figure placement to htbp
\makeatletter
\def\fps@figure{htbp}
\makeatother

\usepackage{float}
\usepackage{tabularray}
\usepackage[normalem]{ulem}
\usepackage{graphicx}
\UseTblrLibrary{booktabs}
\UseTblrLibrary{rotating}
\UseTblrLibrary{siunitx}
\NewTableCommand{\tinytableDefineColor}[3]{\definecolor{#1}{#2}{#3}}
\newcommand{\tinytableTabularrayUnderline}[1]{\underline{#1}}
\newcommand{\tinytableTabularrayStrikeout}[1]{\sout{#1}}
\KOMAoption{captions}{tableheading}
\usepackage{float}
\floatplacement{table}{H}
\floatplacement{figure}{H}
\makeatletter
\@ifpackageloaded{caption}{}{\usepackage{caption}}
\AtBeginDocument{%
\ifdefined\contentsname
  \renewcommand*\contentsname{Tabla de contenidos}
\else
  \newcommand\contentsname{Tabla de contenidos}
\fi
\ifdefined\listfigurename
  \renewcommand*\listfigurename{Listado de Figuras}
\else
  \newcommand\listfigurename{Listado de Figuras}
\fi
\ifdefined\listtablename
  \renewcommand*\listtablename{Listado de Tablas}
\else
  \newcommand\listtablename{Listado de Tablas}
\fi
\ifdefined\figurename
  \renewcommand*\figurename{Figura}
\else
  \newcommand\figurename{Figura}
\fi
\ifdefined\tablename
  \renewcommand*\tablename{Tabla}
\else
  \newcommand\tablename{Tabla}
\fi
}
\@ifpackageloaded{float}{}{\usepackage{float}}
\floatstyle{ruled}
\@ifundefined{c@chapter}{\newfloat{codelisting}{h}{lop}}{\newfloat{codelisting}{h}{lop}[chapter]}
\floatname{codelisting}{Listado}
\newcommand*\listoflistings{\listof{codelisting}{Listado de Listados}}
\makeatother
\makeatletter
\makeatother
\makeatletter
\@ifpackageloaded{caption}{}{\usepackage{caption}}
\@ifpackageloaded{subcaption}{}{\usepackage{subcaption}}
\makeatother
\ifLuaTeX
\usepackage[bidi=basic]{babel}
\else
\usepackage[bidi=default]{babel}
\fi
\babelprovide[main,import]{spanish}
% get rid of language-specific shorthands (see #6817):
\let\LanguageShortHands\languageshorthands
\def\languageshorthands#1{}
\ifLuaTeX
  \usepackage{selnolig}  % disable illegal ligatures
\fi
\usepackage{bookmark}

\IfFileExists{xurl.sty}{\usepackage{xurl}}{} % add URL line breaks if available
\urlstyle{same} % disable monospaced font for URLs
\hypersetup{
  pdftitle={Resolución},
  pdfauthor={María Varela Oyola},
  pdflang={es},
  colorlinks=true,
  linkcolor={blue},
  filecolor={Maroon},
  citecolor={Blue},
  urlcolor={Blue},
  pdfcreator={LaTeX via pandoc}}

\title{Resolución}
\author{María Varela Oyola}
\date{17/10/2025}

\begin{document}
\maketitle

\renewcommand*\contentsname{Tabla de contenidos}
{
\hypersetup{linkcolor=}
\setcounter{tocdepth}{4}
\tableofcontents
}
\pagebreak

\section{Previo al planteamiento de los
problemas}\label{previo-al-planteamiento-de-los-problemas}

Implementamos las funciones de incertidumbre a utilizar

\begin{Shaded}
\begin{Highlighting}[]
\FunctionTok{source}\NormalTok{(}\StringTok{"teoriadecision\_funciones\_incertidumbre.R"}\NormalTok{)}
\end{Highlighting}
\end{Shaded}

Cargamos la librería tinytable para presentar las tablas en un mejor
formato

\begin{Shaded}
\begin{Highlighting}[]
\FunctionTok{library}\NormalTok{(tinytable)}
\end{Highlighting}
\end{Shaded}

\begin{verbatim}
Warning: package 'tinytable' was built under R version 4.3.3
\end{verbatim}

\pagebreak

\section{Problema 1}\label{problema-1}

Aplicar los criterios de decisión bajo incertidumbre al problema cuya
matriz de valores numéricos viene dada en la siguiente tabla:

\begin{longtable}[]{@{}llllll@{}}
\caption{Donde ei indica los estados de la naturaleza (desde i =
1,\ldots,5) y di indica las diferentes alternativas / decisiones a tomar
(desde i = 1,\ldots,4)}\tabularnewline
\toprule\noalign{}
& e1 & e2 & e3 & e4 & e5 \\
\midrule\noalign{}
\endfirsthead
\toprule\noalign{}
& e1 & e2 & e3 & e4 & e5 \\
\midrule\noalign{}
\endhead
\bottomrule\noalign{}
\endlastfoot
d1 & 90 & 40 & 60 & 40 & 75 \\
d2 & 50 & 70 & 55 & 65 & 60 \\
d3 & 20 & 80 & 90 & 75 & 50 \\
d4 & 70 & 45 & 40 & 95 & 30 \\
\end{longtable}

Considera el problema primero de beneficios (favorable) y posteriormente
de costos (desfavorable).

\subsection{Creación de la matriz}\label{creaciuxf3n-de-la-matriz}

\begin{Shaded}
\begin{Highlighting}[]
\CommentTok{\# Creamos la tabla con la función crea.tablaX}
\NormalTok{tabla1 }\OtherTok{=} \FunctionTok{crea.tablaX}\NormalTok{(}\FunctionTok{c}\NormalTok{(}\DecValTok{90}\NormalTok{,}\DecValTok{40}\NormalTok{, }\DecValTok{60}\NormalTok{, }\DecValTok{40}\NormalTok{, }\DecValTok{75}\NormalTok{,}
                       \DecValTok{50}\NormalTok{, }\DecValTok{70}\NormalTok{, }\DecValTok{55}\NormalTok{, }\DecValTok{65}\NormalTok{, }\DecValTok{60}\NormalTok{,}
                       \DecValTok{20}\NormalTok{, }\DecValTok{80}\NormalTok{, }\DecValTok{90}\NormalTok{, }\DecValTok{75}\NormalTok{, }\DecValTok{50}\NormalTok{, }
                       \DecValTok{70}\NormalTok{, }\DecValTok{45}\NormalTok{, }\DecValTok{40}\NormalTok{, }\DecValTok{95}\NormalTok{, }\DecValTok{30}\NormalTok{), }\AttributeTok{numalternativas =} \DecValTok{4}\NormalTok{, }\AttributeTok{numestados =} \DecValTok{5}\NormalTok{)}

\CommentTok{\# Convertimos la tabla en un data.frame}
\NormalTok{tabla\_1 }\OtherTok{=} \FunctionTok{as.data.frame}\NormalTok{(tabla1)}

\CommentTok{\# Presentamos la tabla de valores}
\FunctionTok{tt}\NormalTok{(tabla\_1, }\AttributeTok{rownames =}\NormalTok{ T)}
\end{Highlighting}
\end{Shaded}

\begin{table}
\centering
\begin{tblr}[         %% tabularray outer open
]                     %% tabularray outer close
{                     %% tabularray inner open
colspec={Q[]Q[]Q[]Q[]Q[]Q[]},
}                     %% tabularray inner close
\toprule
rowname & e1 & e2 & e3 & e4 & e5 \\ \midrule %% TinyTableHeader
d1 & 90 & 40 & 60 & 40 & 75 \\
d2 & 50 & 70 & 55 & 65 & 60 \\
d3 & 20 & 80 & 90 & 75 & 50 \\
d4 & 70 & 45 & 40 & 95 & 30 \\
\bottomrule
\end{tblr}
\end{table}

\subsection{Resolución de beneficios}\label{resoluciuxf3n-de-beneficios}

\subsection{Resolución de costes}\label{resoluciuxf3n-de-costes}

\pagebreak

\section{Problema 2}\label{problema-2}

Miguel está pensando en contratar un servicio de Internet para su hogar.
Ha consultado varias compañías y le han dado los siguientes informes:

\begin{itemize}
\item
  Con Digi debe pagar un cuota mensual de 25€, pero la velocidad máxima
  solo estará disponible después de 6 meses; hasta entonces, solo tendrá
  la mitad de velocidad contratada.
\item
  Con O2 puede disfrutar de la velocidad completa desde el primer mes,
  pero deberá pagar una cuota de 50€ al mes.
\item
  Con Vodafone debe pagar 35€ al mes, lo que garantiza la velocidad
  completa, excepto en horas puntas, cuando la velocidad se reduce.
\end{itemize}

Miguel trabaja desde casa y necesita Internet para videollamadas
frecuentes y descargas grandes de archivos. Supongamos que el coste
asociado a la velocidad insuficiente son:

\begin{itemize}
\item
  Reducción en productividad y retrasos en descargas: 400€
\item
  Problemas en videollamadas y reuniones importantes: 1000€
\end{itemize}

Miguel quiere evaluar cuál sería el coste total del primer año,
considerando que podría tener meses de alta demanda de trabajo o meses
de baja demanda, y cuál sería la mejor compañía a elegir.

\subsection{Creación de la matriz}\label{creaciuxf3n-de-la-matriz-1}



\end{document}
